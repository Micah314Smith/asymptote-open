
\documentclass[12pt]{article}
\usepackage{asymptote}
\begin{document}
\begin{asydef}
// Global definitions can be put here.
\end{asydef}

Here is a figure produced with Asymptote, drawn to width 5cm:

\begin{center}
\begin{asy}
size(5cm,0);
pen colour1=red;
pen colour2=green;

pair z0=(0,0);
pair z1=(-1,0);
pair z2=(1,0);
real r=1.5;
guide c1=circle(z1,r);
guide c2=circle(z2,r);
fill(c1,colour1);
fill(c2,colour2);

picture intersection=new picture;
fill(intersection,c1,colour1+colour2);
clip(intersection,c2);

add(intersection);

draw(c1);
draw(c2);

label("$A$",z1);
label("$B$",z2);

pair z=(0,-2);
real m=3;
margin BigMargin=Margin(0,m*dot(unit(z1-z),unit(z0-z)));

draw("$A\cap B$",conj(z)--z0,0,Arrow,BigMargin);
draw("$A\cup B$",z--z0,0,Arrow,BigMargin);
draw(z--z1,Arrow,Margin(0,m));
draw(z--z2,Arrow,Margin(0,m));

shipout(bbox(0.25cm));
\end{asy}
\end{center}

Each graph is drawn in its own environment. One can specify the width
and height to \LaTeX\ explicitly:

\begin{center}
\begin{asy}[3cm,0]
guide center = (0,1){W}..tension 0.8..(0,0){(1,-.5)}..tension 0.8..{W}(0,-1); 

draw((0,1)..(-1,0)..(0,-1));
filldraw(center{E}..{N}(1,0)..{W}cycle);
fill(circle((0,0.5),0.125),white);
fill(circle((0,-0.5),0.125));
\end{asy}
\end{center}

The default width is the full line width:

\begin{center}
\begin{asy}
import graph;

real f(real x) {return sqrt(x);}
pair F(real x) {return (x,f(x));}

real g(real x) {return -sqrt(x);}
pair G(real x) {return (x,g(x));}

guide p=(0,0)--graph(f,0,1,Spline)--(1,0);
fill(p--cycle,lightgray);
draw(p);
draw((0,0)--graph(g,0,1,Spline)--(1,0),dotted);

real x=0.5;
pair c=(4,0);

transform T=xscale(0.5);
draw((2.695,0),T*arc(0,0.30cm,20,340),ArcArrow);
fill(shift(c)*T*circle(0,-f(x)),red+white);
draw(F(x)--c+(0,f(x)),dashed+red);
draw(G(x)--c+(0,g(x)),dashed+red);

labeldot((1,1));
arrow("$y=\sqrt{x}$",F(0.7),N);

arrow((3,0.5*f(x)),W,1cm,red);
arrow((3,-0.5*f(x)),W,1cm,red);

xaxis(0,c.x,"$x$",dashed);
yaxis("$y$");

draw("$r$",(x,0)--F(x),E,red,Arrows,BeginBar,PenMargins);
draw("$r$",(x,0)--G(x),E,red,Arrows,PenMargins);
draw("$r$",c--c+(0,f(x)),Arrow,PenMargin);
dot(c);
\end{asy}
\end{center}

\end{document}
