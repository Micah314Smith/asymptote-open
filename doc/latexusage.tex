\documentclass[12pt]{article}

% Use this form to include eps (latex) or pdf (pdflatex) files:
\usepackage{asymptote}

% Use this form with latex or pdflatex to include inline LaTeX code:
%\usepackage[inline]{asymptote}

% Enable this line to produce pdf hyperlinks with latex:
%\usepackage[hypertex]{hyperref}

% Enable this line to produce pdf hyperlinks with pdflatex:
%\usepackage[pdftex]{hyperref}

\begin{document}

\begin{asydef}
// Global Asymptote definitions can be put here.
usepackage("bm");
texpreamble("\def\v#1{\bm{#1}}");
// Uncomment the next line to show the toolbar
// settings.toolbar=true;
\end{asydef}

Here is a venn diagram produced with Asymptote, drawn to width 4cm:

\def\A{A}
\def\B{\v{B}}

%\begin{figure}
\begin{center}
\begin{asy}
size(4cm,0);
pen colour1=red;
pen colour2=green;

pair z0=(0,0);
pair z1=(-1,0);
pair z2=(1,0);
real r=1.5;
path c1=circle(z1,r);
path c2=circle(z2,r);
fill(c1,colour1);
fill(c2,colour2);

picture intersection=new picture;
fill(intersection,c1,colour1+colour2);
clip(intersection,c2);

add(intersection);

draw(c1);
draw(c2);

//draw("$\A$",box,z1);              // Requires [inline] package option.
//draw(Label("$\B$","$B$"),box,z2); // Requires [inline] package option.
draw("$A$",box,z1);            
draw("$\v{B}$",box,z2);

pair z=(0,-2);
real m=3;
margin BigMargin=Margin(0,m*dot(unit(z1-z),unit(z0-z)));

draw(Label("$A\cap B$",0),conj(z)--z0,Arrow,BigMargin);
draw(Label("$A\cup B$",0),z--z0,Arrow,BigMargin);
draw(z--z1,Arrow,Margin(0,m));
draw(z--z2,Arrow,Margin(0,m));

shipout(bbox(0.25cm));
\end{asy}
%\caption{Venn diagram}\label{venn}
\end{center}
%\end{figure}

Each graph is drawn in its own environment. One can specify the width
and height to \LaTeX\ explicitly. This 3D example can be viewed
interactively either with Adobe Reader or Asymptote's fast OpenGL-based
renderer:
\begin{center}
\begin{asy}[0,4cm]
import three;
if(settings.render < 0) settings.render=8;

draw(unitcube,blue);
label("$V-E+F=2$",(0,1,0.5),3Y,blue+fontsize(17));
\end{asy}
\end{center}

One can also scale the figure to the full line width:
\begin{center}
\begin{asy}[\the\linewidth]
pair z0=(0,0);
pair z1=(2,0);
pair z2=(5,0);
pair zf=z1+0.75*(z2-z1);

draw(z1--z2);
dot(z1,red+0.15cm);
dot(z2,darkgreen+0.3cm);
label("$m$",z1,1.2N,red);
label("$M$",z2,1.5N,darkgreen);
label("$\hat{\ }$",zf,0.2*S,fontsize(24)+blue);

pair s=-0.2*I;
draw("$x$",z0+s--z1+s,N,red,Arrows,Bars,PenMargins);
s=-0.5*I;
draw("$\bar{x}$",z0+s--zf+s,blue,Arrows,Bars,PenMargins);
s=-0.95*I;
draw("$X$",z0+s--z2+s,darkgreen,Arrows,Bars,PenMargins);
\end{asy}
\end{center}
\end{document}
